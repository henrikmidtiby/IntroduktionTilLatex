\documentclass{article}
\usepackage[utf8]{inputenc}
\usepackage{amsmath}

\begin{document}

Ifølge pythagoras er sidelængden af en retvinklet trekants hypotenuse $c$ givet ud fra 
kateterne ($a$ og $b$). Nogle heltallige eksempler er givet i tabel \ref{tabPythagoras}.
De er genereret ud fra de to hele tal $n$ og $m$ ud fra nedenstående formel.

\begin{align}
(a, b, c) = (m^2 – n^2, 2 \cdot m \cdot n, m^2 + n^2) 
\end{align}


\begin{table}
\centering
\begin{tabular}{lr|rrr}
$n$ & $m$ & $a$ & $b$ & $c$ \\
\hline
1 & 2 	& 3 		& 4 		& 5	\\
1 & 3 	& 8 		& 6 		& 10	\\
2 & 3 	& 5 		& 12 	& 13	\\
1 & 4 	& 15 	& 8 		& 17	\\
2 & 4 	& 12 	& 16 	& 20	\\
3 & 4 	& 7 		& 24 	& 25	\\
\end{tabular}
\caption{Pythagorære tripler}
\label{tabPythagoras}
\end{table}



\end{document}